% !TEX root = Tesi.tex
\chead{}
\chapter{Our approach to modeling}

After summarily describing three different streams of real usage data obtained both from the physical and the virtual world in the last chapter, we now focus on expanding those representations in a more detailed way with the help of the Model Driven Engineering techniques briefly described in \ref{model-driven-techniques}. 
Concretely, the first two section objectives are to illustrate the defining languages (metamodels) for both the real usage data and the eCommerce platform interactions used in the previous examples and generate actual models based upon them representing the very same information.
Finally, in the last section of this chapter, we will be using the very same generation for updating the previously instantiated models leveraging model transformation techniques based upon usage pattern detection resulting from the Big Data analysis.

\section{Real usage data modeling}

\subsection{Metamodel}

The representation of the real usage data starts from the definition of the metamodel which defines the languages and processes from which to form a model without making statements about its content. In fact, a metamodel is itself a model that is used to describe another model using a modeling language and at a different level of abstraction.  

The figure in \ref{fig:real-usage-data-metamodel-diagram} describes the processed metamodel as a UML Class Diagram accordingly to the data retrieved for our real usage data analysis.

\vspace{0.5cm}
\begin{figure}[H]
  \centering
    \includegraphics[height=19cm]{images/diagrams/RealUsageDataMetamodel.jpg}
  \caption{Real Usage Data Metamodel Diagram Class}
  \label{fig:real-usage-data-metamodel-diagram}
\end{figure}
\vspace{0.5cm}

\newpage
\subsection{Model}
\label{real-usage-data-model}
The RealUsageData metamodel defined above allow us to create dynamic instances which precisely map the real usage data collected from the web mining process and the IoT devices tracking. Figure \ref{fig:real-usage-data-model} illustrates this processed model in its eCore representation form in Eclipse and it is followed by the corrisponding XMI file content.

\vspace{0.5cm}
\begin{figure}[H]
  \centering
    \includegraphics[height=12cm]{images/diagrams/RealUsageDataModel.png}
  \caption{Real Usage Data Model}
  \label{fig:real-usage-data-model}
\end{figure}
\vspace{0.5cm}

\lstset{language=XML}
\begin{lstlisting} 
<?xml version="1.0" encoding="UTF-8"?>
<RealUsageData:Dataset
    xmi:version="2.0"
    xmlns:xmi="http://www.omg.org/XMI"
    xmlns:xsi="http://www.w3.org/2001/XMLSchema-instance"
    xmlns:RealUsageData="RealUsageData"
    xsi:schemaLocation="RealUsageData ../metamodels/RealUsageData.ecore"
    ID="1" name="Main">
  <data xsi:type="RealUsageData:WebData"
      ID="1"
      name="AccessLog"
      userID="3045678"
      date="2017-11-29T17:06:49.000+0100"
      viewContainer="Homepage"
      viewComponent="TopMenu"
      eventType="click"
      parameterBindingGroup="Category/5"
      logEntry="GET /men.html / 200 0 - 29505"/>
  <data xsi:type="RealUsageData:WebData"
      ID="2"
      name="AccessLog"
      userID="3045678"
      date="2017-11-29T17:07:04.000+0100"
      viewContainer="Category #5"
      viewComponent="CategoryList"
      eventType="click"
      parameterBindingGroup="Category/15"
      logEntry="GET /men/shirts.html 200 0 - 29505"/>
  <data xsi:type="RealUsageData:WebData"
      ID="3"
      name="AccessLog"
      userID="3045678"
      date="2017-11-29T07:08:40.000+0100"
      viewContainer="Category #15"
      viewComponent="ProductList"
      eventType="click"
      parameterBindingGroup="Product/404"
      logEntry="GET /men/shirts/plaid-cotton-shirt-476.html 200 0 - 29505"/>
  <data xsi:type="RealUsageData:WebData"
      ID="4"
      name="AccessLog"
      userID="3045678"
      date="2017-12-04T06:37:15.000+0100"
      viewContainer="Product #404"
      viewComponent="RelatedProductList"
      eventType="click"
      parameterBindingGroup="Product/413"
      logEntry="GET /core-striped-sport-shirt-551.html 200 0 - 29505"/>
  <data xsi:type="RealUsageData:WebData"
      ID="5"
      name="AccessLog"
      userID="3045678"
      date="2017-12-04T06:37:21.000+0100"
      viewContainer=""
      viewComponent=""
      eventType="backButton"
      parameterBindingGroup=""
      logEntry="GET /men/shirts/plaid-cotton-shirt-476.html 200 0 - 29505"/>
  <data xsi:type="RealUsageData:WebData"
      ID="6"
      name="AccessLog"
      userID="3045678"
      date="2017-12-04T06:38:06.000+0100"
      viewContainer="Product #404"
      viewComponent="TopMenu"
      eventType="click"
      parameterBindingGroup="Category/16"
      logEntry="GET /men/tees-knits-and-polos.html 200 0 - 29505"/>
  <data xsi:type="RealUsageData:WebData"
      ID="7"
      name="AccessLog"
      userID="3045678"
      date="2017-12-04T06:38:20.000+0100"
      viewContainer="Category #16"
      viewComponent="TopSearch"
      eventType="submit"
      parameterBindingGroup="SearchText/blazer"
      logEntry="GET /catalogsearch/result/?q=blazer 200 0 - 29505"/>
  <data xsi:type="RealUsageData:WebData"
      ID="8"
      name="AccessLog"
      userID="3045678"
      date="2017-12-04T06:38:20.000+0100"
      viewContainer="Search Results"
      viewComponent="ProductList"
      eventType="click"
      parameterBindingGroup="Product/407"
      logEntry="GET /stretch-cotton-blazer-587.html 200 0 - 29505"/>
  <data xsi:type="RealUsageData:ProximityData"
      ID="9"
      name="Session"
      userID="3045678"
      date="2018-02-21T18:16:07.000+0100"
      storeId="8784"
      storeLabel="Madison1"
      sessionId="89376f84-065b -11e8- ba89-0ed5f89f718b"
      sessionDuration="345">
    <sessionRegions
        regionId="156"
        regionLabel="store-entrance"
        detectionCount="2"
        maxSecondsInRegion="5"
        firstDetectionTimeStamp="2018-02-21T18:09:07.000+0100"
        lastDetectionTimeStamp="2018-02-21T18:16:02.000+0100">
      <beaconData
          uuid="0686a88e - fed6-11e7-8be5-0ed5f89f718b"
          majorId="2553"
          minorId="79"/>
    </sessionRegions>
    <sessionRegions
        regionId="645"
        regionLabel="shoes"
        detectionCount="1"
        maxSecondsInRegion="24"
        maxProximity="near"
        firstDetectionTimeStamp="2018-02-21T18:09:20.000+0100"
        lastDetectionTimeStamp="2018-02-21T18:09:20.000+0100">
      <beaconData
          uuid="0686a88e-fed6-11e7-8be5-0ed5f89f718b"
          majorId="19029"
          minorId="49"/>
    </sessionRegions>
    <sessionRegions
        regionId="6875"
        regionLabel="jewelry"
        detectionCount="1"
        maxSecondsInRegion="15"
        maxProximity="far"
        firstDetectionTimeStamp="2018-02-21T18:10:15.000+0100"
        lastDetectionTimeStamp="2018-02-21T18:10:15.000+0100">
      <beaconData
          uuid="0686a88e-fed6-11e7-8be5-0ed5f89f718b"
          majorId="38415"
          minorId="59"/>
    </sessionRegions>
    <sessionRegions
        regionId="2563"
        regionLabel="blazers"
        detectionCount="1"
        maxSecondsInRegion="195"
        maxProximity="immediate"
        firstDetectionTimeStamp="2018-02-21T18:11:01.000+0100"
        lastDetectionTimeStamp="2018-02-21T18:11:01.000+0100">
      <beaconData
          uuid="0686a88e-fed6-11e7-8be5-0ed5f89f718b"
          majorId="25911"
          minorId="27"/>
    </sessionRegions>
    <sessionRegions
        regionId="456"
        regionLabel="tees-knits-polos"
        detectionCount="1"
        maxSecondsInRegion="10"
        maxProximity="immediate"
        firstDetectionTimeStamp="2018-02-21T18:14:56.000+0100"
        lastDetectionTimeStamp="2018-02-21T18:14:56.000+0100">
      <beaconData
          uuid="0686a88e-fed6-11e7-8be5-0ed5f89f718b"
          majorId="42037"
          minorId="36"/>
    </sessionRegions>
    <sessionRegions
        regionId="998"
        regionLabel="bags-and-luggage"
        detectionCount="1"
        maxSecondsInRegion="7"
        maxProximity="far"
        firstDetectionTimeStamp="2018-02-21T18:15:12.000+0100"
        lastDetectionTimeStamp="2018-02-21T18:15:12.000+0100">
      <beaconData
          uuid="0686a88e-fed6-11e7-8be5-0ed5f89f718b"
          majorId="37931"
          minorId="85"/>
    </sessionRegions>
  </data>
  <data xsi:type="RealUsageData:ActionData"
      ID="10"
      name="Session"
      userID="3045678"
      date="2018-02-22T15:27:09.000+0100"
      storeId="8784"
      storeLabel="Madison1"
      sessionId="89376f84-065b-11e8-ba89-0ed5f89f718b"
      sessionDuration="456">
    <sessionActions
        userAgent="iPhone 6S">
      <scannedItems
          barcode="042100005264"
          name="Elizabeth Knit Top-Red-S"
          sku="wbk012c-Red-S"/>
      <scannedItems
          barcode="042100005931"
          name="Plaid Cotton Shirt-Khaki-L"
          sku="msj006c-Khaki-L"/>
      <scannedItems
          barcode="042100007717"
          name="Broad St Saddle Shoes"
          sku="shm00110"/>
    </sessionActions>
  </data>
</RealUsageData:Dataset>
\end{lstlisting}

\section{eCommerce application modeling}

As per briefly introduced in \ref{the-ifml-language}, IFML is used to design platform independent-level models which can be used to define the interactions between the users of an application and the application itself. 
At its core, IFML is meant to be flexible and straightforward, but perhaps more importantly, the language is intended to be abstract to provide the possibility of defining the main traits of an application’s front end making as few visual commitments as possible. Furthermore, its extendibility  allows modelers and designers to specialize specific components to enrich the semantics of their models and make the diagrams more readable.

In fact, the models generated using the Interaction Flow language describe the user interface components required at the front-end of the application, without specifying layout details of these elements enhancing the separation of concerns among developers and UX designers where the latter build the user interface accordingly to an interaction flow model. Besides defining components of the User Interface, these models explain how data flows among different sections of the application upon triggering events and introduces the business logic carried out using this data.

\subsection{IFML Metamodel}

The IFML metamodel is organized into three packages: the Core package, the Extension package and the DataTypes package. The Core package contains the concepts that build up the interaction infrastructure of the language concerning InteractionFlowElements, InteractionFlows and Parameters. The Extension package extends the Core package components with more complex behaviors. The DataTypes package contains the custom data types defined by IFML.

\vspace{0.5cm}
\begin{figure}[H]
  \centering
    \includegraphics[width=14cm]{images/diagrams/ifml-ecore.png}
  \caption{IFML Ecore representation}
  \label{fig:ifml-ecore-representation}
\end{figure}
\vspace{0.5cm}


By using the primitive data types from the UML metamodel and a UML representation for the IFML Domain Model, the IFML metamodel specifies a set of UML metaclasses as the foundation for the IFML metaclasses.

The following is the structure of the high-level representation of the IFML metamodel and its areas of concern:

\begin{itemize}
  \item IFML Model
  \item Interaction Flow Model
  \item Interaction Flow Elements
  \item View Elements
  \item Events
  \item Specific Events and View Components
  \item Parameters
  \item Expressions
  \item ContentBinding
\end{itemize}

Figure \ref{fig:simple-ifml-core-model} shows an excerpt of the IFML metamodel. As can be seen, IFMLModel is the top-level container of all the model elements and represents an IFML model. It contains an InteractionFlowModel which is the user view of an application, a DomainModel represented in UML and optionally ViewPoints. The concepts extending ViewContainer, ViewComponets, ViewComponentPart, and ViewElementEvent represent the visual elements of an IFML model.

\vspace{0.5cm}
\begin{figure}[H]
  \centering
    \includegraphics[width=12cm]{images/diagrams/ifml-metamodel.png}
  \caption{Simple Ecore model of an IFML subset.}
  \label{fig:simple-ifml-core-model}
\end{figure}
\vspace{0.5cm}

\subsection{Model}

As per mentioned in the last subsection, interaction flow models are described using the Interaction Flow Modelling Language and, together with the domain model and optionally viewpoints, they form the core of the IFML model.

Essentially, the domain model objective is offering to the interaction flow references about the content available. An example of a domain model for an e-commerce website is given in figure \ref{fig:domain-model-uml-ecommerce}   

\vspace{0.5cm}
\begin{figure}[H]
  \centering
    \includegraphics[width=12cm]{images/diagrams/domain-model-uml-ecommerce.png}
  \caption{Domain Model UML Class Diagram ecommerce example.}
  \label{fig:domain-model-uml-ecommerce}
\end{figure}
\vspace{0.5cm}

Although some partial IFML model representations for the Madison Island eCommerce platform have been already summarily introduced in \ref{navigational-modeling-for-the-web}, in this subsection we examine the most important ones in more detail and with a more global approach not strictly related to the navigational modeling. The final goal is to model, taking advantage of the IFML metamodel described just above, an IFML model which would represent the main pages and interactions of the website.

\subsubsection{Homepage}

\vspace{0.5cm}
\begin{figure}[H]
  \centering
    \includegraphics[width=11cm]{images/diagrams/before/desktop-homepage.png}
  \caption{Homepage Desktop Version}
  \label{fig:desktop-before-homepage}
\end{figure}
\vspace{0.5cm}

\begin{figure}[H]
  \centering
    \includegraphics[height=9cm]{images/diagrams/before/ifml-homepage.png}
  \caption{Homepage IFML Diagram}
  \label{fig:ifml-before-homepage}
\end{figure}
\vspace{0.5cm}

The Madison Island Interaction Model for the Homepage (Figure \ref{fig:desktop-before-homepage} and \ref{fig:ifml-before-homepage}) is composed by a parent IFMLWindow element which contains three children elements: respectively an IFMLWindow for the Highlighted Categories Carousel, a Detail View Component for the Homepage promos CMS Block and a \textbf{"List View Component"} for the New Products sections bound to the Product Entity of the domain model. The HighlightedCategoriesCarousel Window is in \textbf{"XOR mode"} representing three possible scenarios for the category to promote with the highest priority within the carousel mechanism. Each data binding within all these view containers is limited by a \textbf{"Conditional Expressions"} defining the instance of the content to show. 

The following is an extract from the IFML Model for the first \textit{HighlighedCategoryBanner View Container} element: 


\lstset{language=XML}
\begin{lstlisting} 
      <viewElements xsi:type="ext:Details"  name="CategoryDetail">
        <parameters  name="Highlighted Category #1" direction="inout">
          <constraints  language="SQL" body="Category.ID=18"/>
          <type xsi:type="uml:Class" href="model.uml#__W1boJ6PEeGdnpRmAZh-dQ"/>
        </parameters>
        <viewElementEvents xsi:type="ext:OnSelectEvent"  name="Details" viewElement="//@interactionFlowModel/@interactionFlowModelElements.0/@viewElements.0/@viewElements.0">
          <outInteractionFlows xsi:type="core:NavigationFlow"  targetInteractionFlowElement="//@interactionFlowModel/@interactionFlowModelElements.6">
            <parameterBindingGroup >
              <parameterBindings  sourceParameter="//@interactionFlowModel/@interactionFlowModelElements.0/@viewElements.0/@viewElements.0/@viewElements.0/@parameters.0" targetParameter="//@interactionFlowModel/@interactionFlowModelElements.6/@parameters.0"/>
            </parameterBindingGroup>
          </outInteractionFlows>
        </viewElementEvents>
        <viewComponentParts xsi:type="core:DataBinding"  name="Category" uniformResourceIdentifier="">
          <subViewComponentParts xsi:type="core:ConditionalExpression"  language="SQL" body="Category.ID=18" name="Eyewear"/>
        </viewComponentParts>
        <viewComponentParts xsi:type="core:VisualizationAttribute"  name="Image" featureConcept="//@domainModel/@domainElements.4"/>
      </viewElements>
    </viewElements>
\end{lstlisting}

The above snippet belongs to a more complex IFML model hierarchy as shown in \ref{fig:ifml-before-hierarchy-homepage}.

\vspace{0.5cm}
\begin{figure}[H]
  \centering
    \includegraphics[width=10cm]{images/diagrams/before/ifml-hierarchy-homepage.png}
  \caption{Expanded Interaction Flow Homepage Model Hierarchy}
  \label{fig:ifml-before-hierarchy-homepage}
\end{figure}
\vspace{0.5cm}

\subsubsection{Category Page}

\vspace{0.5cm}
\begin{figure}[H]
  \centering
  \subfloat[Display Mode PAGE]{{\includegraphics[width=7cm]{images/diagrams/before/desktop-category1.png} }}%
  \qquad
  \subfloat[Display Mode PRODUCTS]{{\includegraphics[width=7cm]{images/diagrams/before/desktop-category2.png} }}%
  \caption{Category Desktop Versions}%
  \label{fig:desktop-before-category}%
\end{figure}
\vspace{0.5cm}

\begin{figure}[H]
  \centering
    \includegraphics[height=9cm]{images/diagrams/before/ifml-category.png}
  \caption{Category IFML Diagram}
  \label{fig:ifml-before-category}
\end{figure}
\vspace{0.5cm}

The Madison Island Interaction Model for the Category Page (Figure \ref{fig:desktop-before-category} and \ref{fig:ifml-before-category}) is composed by a parent IFMLWindow element in \textbf{"XOR mode"} which presents information about the current category on the top of the page. Depending on the display mode property for the Category Entity the user can be presented with two different View Containers that are activated using different \textbf{"Activation Expressions"} based on the property itself. Whilst the first scenario presents a Details type View Component attached to a linked CMS Block ,the second option shows two children view components representing both the filter sidebar and the products listing section with this last one having multiple \textbf{"Visualization Attribute"} children indicating the user is presented with a thumbnail, a name and a price for each product belonging to the current category.

The following is an extract from the IFML Model for the first \textit{Category Products List} element we just described: 


\lstset{language=XML}
\begin{lstlisting} 
    <viewElements xsi:type="ext:List"  name="Category Products">
    <viewElementEvents xsi:type="ext:OnSelectEvent"  name="Product Selected" viewElement="//@interactionFlowModel/@interactionFlowModelElements.6/@viewElements.1/@viewElements.0">
      <outInteractionFlows xsi:type="core:NavigationFlow"  targetInteractionFlowElement="//@interactionFlowModel/@interactionFlowModelElements.1">
        <parameterBindingGroup >
          <parameterBindings  sourceParameter="//@interactionFlowModel/@interactionFlowModelElements.1/@parameters.0" targetParameter="//@interactionFlowModel/@interactionFlowModelElements.1/@parameters.0"/>
        </parameterBindingGroup>
      </outInteractionFlows>
    </viewElementEvents>
    <viewComponentParts xsi:type="core:DataBinding"  name="Product" domainConcept="//@domainModel/@domainElements.3">
      <conditionalExpression  language="SQL" body="Category IN Product.Categories" name="Category Products"/>
    </viewComponentParts>
    <viewComponentParts xsi:type="core:VisualizationAttribute"  name="Image" featureConcept="//@domainModel/@domainElements.7"/>
    <viewComponentParts xsi:type="core:VisualizationAttribute"  name="Name" featureConcept="//@domainModel/@domainElements.8"/>
    <viewComponentParts xsi:type="core:VisualizationAttribute"  name="Price" featureConcept="//@domainModel/@domainElements.9"/>
  </viewElements>
  <viewElements xsi:type="core:ViewComponent"  name="Product Filter Sidebar">
    <viewComponentParts xsi:type="core:DataBinding"  name="Category"/>
  </viewElements>
</viewElements>
\end{lstlisting}

The full expanded model hierarchy for the IFML Window Category element is shown in \ref{fig:ifml-before-hierarchy-category}.


\vspace{0.5cm}
\begin{figure}[H]
  \centering
    \includegraphics[width=13cm]{images/diagrams/before/ifml-hierarchy-category.png}
  \caption{Expanded Interaction Flow Category Model Hierarchy}
  \label{fig:ifml-before-hierarchy-category}
\end{figure}
\vspace{0.5cm}


\subsubsection{Product Page}

\vspace{0.5cm}
\begin{figure}[H]
  \centering
    \includegraphics[width=10cm]{images/diagrams/before/desktop-product.png}
  \caption{Product Page Desktop Version}
  \label{fig:desktop-before-product}
\end{figure}

\vspace{0.5cm}
\begin{figure}[H]
  \centering
    \includegraphics[height=9cm]{images/diagrams/before/ifml-product.png}
  \caption{Product Page IFML Diagram}
  \label{fig:ifml-before-product}
\end{figure}
\vspace{0.5cm}

The Madison Island Interaction Model for the Product page (Figure \ref{fig:desktop-before-product} and \ref{fig:ifml-before-product}) is mainly built with a single IFMLWindow element containing different types of \textbf{"View Components"} with the main one being a Detail one bound to the current product data entity. The other two elements are the single \textbf{"Form"} view component descibing the Add to Cart section and its possible interactions and a \textbf{"List"} view component holding the information for the Related Product widget.

The hierarchy described above has the following form as IFML Model :

\lstset{language=XML}
\begin{lstlisting} 
    <interactionFlowModelElements xsi:type="ext:IFMLWindow"  name="Product" inInteractionFlows="//@interactionFlowModel/@interactionFlowModelElements.1/@viewElements.2/@viewElementEvents.0/@outInteractionFlows.0 //@interactionFlowModel/@interactionFlowModelElements.0/@viewElements.2/@viewElementEvents.0/@outInteractionFlows.0 //@interactionFlowModel/@interactionFlowModelElements.10/@viewElements.0/@viewElementEvents.0/@outInteractionFlows.0 //@interactionFlowModel/@interactionFlowModelElements.6/@viewElements.1/@viewElements.0/@viewElementEvents.0/@outInteractionFlows.0">
      <parameters  name="Product">
        <type xsi:type="uml:Class" href="model.uml#_nyxiEA9LEeiZ14TmPBeBNA"/>
      </parameters>
      <viewElements xsi:type="ext:Details"  name="ProductDetails">
        <viewComponentParts xsi:type="core:DataBinding"  name="Product" uniformResourceIdentifier="">
          <subViewComponentParts xsi:type="core:VisualizationAttribute"  name="Price" featureConcept="//@domainModel/@domainElements.9"/>
          <subViewComponentParts xsi:type="core:VisualizationAttribute"  name="Image" featureConcept="//@domainModel/@domainElements.7"/>
          <subViewComponentParts xsi:type="core:VisualizationAttribute"  name="Name" featureConcept="//@domainModel/@domainElements.8"/>
          <subViewComponentParts xsi:type="core:VisualizationAttribute"  name="Description" featureConcept="//@domainModel/@domainElements.10"/>
        </viewComponentParts>
      </viewElements>
      <viewElements xsi:type="ext:Form"  name="AddToCartForm">
        <viewElementEvents xsi:type="ext:OnSubmitEvent"  name="Add To Cart" viewElement="//@interactionFlowModel/@interactionFlowModelElements.1/@viewElements.1">
          <outInteractionFlows xsi:type="core:NavigationFlow"  targetInteractionFlowElement="//@interactionFlowModel/@interactionFlowModelElements.9">
            <parameterBindingGroup >
              <parameterBindings  sourceParameter="//@interactionFlowModel/@interactionFlowModelElements.1/@viewElements.1/@viewComponentParts.2" targetParameter="//@interactionFlowModel/@interactionFlowModelElements.1/@viewElements.1/@viewComponentParts.2"/>
            </parameterBindingGroup>
          </outInteractionFlows>
        </viewElementEvents>
        <viewComponentParts xsi:type="ext:SelectionField"  name="Color">
          <type xsi:type="uml:PrimitiveType" href="model.uml#_VK2hkJ6QEeGdnpRmAZh-dQ"/>
        </viewComponentParts>
        <viewComponentParts xsi:type="ext:SelectionField"  name="Size">
          <type xsi:type="uml:PrimitiveType" href="model.uml#_VK2hkJ6QEeGdnpRmAZh-dQ"/>
        </viewComponentParts>
        <viewComponentParts xsi:type="ext:SimpleField"  name="Quantity">
          <type xsi:type="uml:PrimitiveType" href="model.uml#_YGTmEJ6QEeGdnpRmAZh-dQ"/>
        </viewComponentParts>
      </viewElements>
      <viewElements xsi:type="ext:List"  name="RelatedProductList">
        <viewElementEvents xsi:type="ext:OnSelectEvent"  name="Product Selected" viewElement="//@interactionFlowModel/@interactionFlowModelElements.1/@viewElements.2">
          <outInteractionFlows xsi:type="core:NavigationFlow"  targetInteractionFlowElement="//@interactionFlowModel/@interactionFlowModelElements.1">
            <parameterBindingGroup >
              <parameterBindings  sourceParameter="//@interactionFlowModel/@interactionFlowModelElements.0/@viewElements.2/@parameters.0" targetParameter="//@interactionFlowModel/@interactionFlowModelElements.1/@parameters.0"/>
            </parameterBindingGroup>
          </outInteractionFlows>
        </viewElementEvents>
        <viewComponentParts xsi:type="core:DataBinding"  name="Product"/>
        <viewComponentParts xsi:type="core:VisualizationAttribute"  name="Image" featureConcept="//@domainModel/@domainElements.7"/>
        <viewComponentParts xsi:type="core:VisualizationAttribute"  name="Name" featureConcept="//@domainModel/@domainElements.8"/>
        <viewComponentParts xsi:type="core:VisualizationAttribute"  name="Price" featureConcept="//@domainModel/@domainElements.9"/>
      </viewElements>
    </interactionFlowModelElements>
\end{lstlisting}

The Model representation of this Product page structure is shown in \ref{fig:ifml-before-hierarchy-product}

\vspace{0.5cm}
\begin{figure}[H]
  \centering
    \includegraphics[width=10cm]{images/diagrams/before/ifml-hierarchy-product.png}
  \caption{Expanded Interaction Flow Product Model Hierarchy}
  \label{fig:ifml-before-hierarchy-product}
\end{figure}
\vspace{0.5cm}

\newpage
\subsubsection{Shopping Cart Page}

\vspace{0.5cm}
\begin{figure}[H]
  \centering
    \includegraphics[width=12cm]{images/diagrams/before/desktop-shoppingcart.png}
  \caption{Shopping Cart Page Desktop Version}
  \label{fig:desktop-before-shoppingcart}
\end{figure}

\vspace{0.5cm}
\begin{figure}[H]
  \centering
    \includegraphics[height=9cm]{images/diagrams/before/ifml-shoppingcart.png}
  \caption{Shopping Cart Page IFML Diagram}
  \label{fig:ifml-before-shoppingcart}
\end{figure}
\vspace{0.5cm}

The Madison Island Interaction Model for the Shopping Cart page (Figure \ref{fig:desktop-before-shoppingcart} and \ref{fig:ifml-before-shoppingcart}) is mainly built with a single IFMLWindow element containing different types of \textbf{"View Components"} with the main one being a Detail one bound to the current product data entity. The other two elements are the single \textbf{"Form"} view component descibing the Add to Cart section and its possible interactions and a \textbf{"List"} view component holding the information for the Related Product widget.

As per shown in \ref{fig:ifml-before-hierarchy-shoppingcart}, the Interaction Flow model representing the product page has the following form:

\lstset{language=XML}
\begin{lstlisting} 
    <interactionFlowModelElements xsi:type="ext:IFMLWindow"  name="Product" inInteractionFlows="//@interactionFlowModel/@interactionFlowModelElements.1/@viewElements.2/@viewElementEvents.0/@outInteractionFlows.0 //@interactionFlowModel/@interactionFlowModelElements.0/@viewElements.2/@viewElementEvents.0/@outInteractionFlows.0 //@interactionFlowModel/@interactionFlowModelElements.10/@viewElements.0/@viewElementEvents.0/@outInteractionFlows.0 //@interactionFlowModel/@interactionFlowModelElements.6/@viewElements.1/@viewElements.0/@viewElementEvents.0/@outInteractionFlows.0">
      <parameters  name="Product">
        <type xsi:type="uml:Class" href="model.uml#_nyxiEA9LEeiZ14TmPBeBNA"/>
      </parameters>
      <viewElements xsi:type="ext:Details"  name="ProductDetails">
        <viewComponentParts xsi:type="core:DataBinding"  name="Product" uniformResourceIdentifier="">
          <subViewComponentParts xsi:type="core:VisualizationAttribute"  name="Price" featureConcept="//@domainModel/@domainElements.9"/>
          <subViewComponentParts xsi:type="core:VisualizationAttribute"  name="Image" featureConcept="//@domainModel/@domainElements.7"/>
          <subViewComponentParts xsi:type="core:VisualizationAttribute"  name="Name" featureConcept="//@domainModel/@domainElements.8"/>
          <subViewComponentParts xsi:type="core:VisualizationAttribute"  name="Description" featureConcept="//@domainModel/@domainElements.10"/>
        </viewComponentParts>
      </viewElements>
      <viewElements xsi:type="ext:Form"  name="AddToCartForm">
        <viewElementEvents xsi:type="ext:OnSubmitEvent"  name="Add To Cart" viewElement="//@interactionFlowModel/@interactionFlowModelElements.1/@viewElements.1">
          <outInteractionFlows xsi:type="core:NavigationFlow"  targetInteractionFlowElement="//@interactionFlowModel/@interactionFlowModelElements.9">
            <parameterBindingGroup >
              <parameterBindings  sourceParameter="//@interactionFlowModel/@interactionFlowModelElements.1/@viewElements.1/@viewComponentParts.2" targetParameter="//@interactionFlowModel/@interactionFlowModelElements.1/@viewElements.1/@viewComponentParts.2"/>
            </parameterBindingGroup>
          </outInteractionFlows>
        </viewElementEvents>
        <viewComponentParts xsi:type="ext:SelectionField"  name="Color">
          <type xsi:type="uml:PrimitiveType" href="model.uml#_VK2hkJ6QEeGdnpRmAZh-dQ"/>
        </viewComponentParts>
        <viewComponentParts xsi:type="ext:SelectionField"  name="Size">
          <type xsi:type="uml:PrimitiveType" href="model.uml#_VK2hkJ6QEeGdnpRmAZh-dQ"/>
        </viewComponentParts>
        <viewComponentParts xsi:type="ext:SimpleField"  name="Quantity">
          <type xsi:type="uml:PrimitiveType" href="model.uml#_YGTmEJ6QEeGdnpRmAZh-dQ"/>
        </viewComponentParts>
      </viewElements>
      <viewElements xsi:type="ext:List"  name="RelatedProductList">
        <viewElementEvents xsi:type="ext:OnSelectEvent"  name="Product Selected" viewElement="//@interactionFlowModel/@interactionFlowModelElements.1/@viewElements.2">
          <outInteractionFlows xsi:type="core:NavigationFlow"  targetInteractionFlowElement="//@interactionFlowModel/@interactionFlowModelElements.1">
            <parameterBindingGroup >
              <parameterBindings  sourceParameter="//@interactionFlowModel/@interactionFlowModelElements.0/@viewElements.2/@parameters.0" targetParameter="//@interactionFlowModel/@interactionFlowModelElements.1/@parameters.0"/>
            </parameterBindingGroup>
          </outInteractionFlows>
        </viewElementEvents>
        <viewComponentParts xsi:type="core:DataBinding"  name="Product"/>
        <viewComponentParts xsi:type="core:VisualizationAttribute"  name="Image" featureConcept="//@domainModel/@domainElements.7"/>
        <viewComponentParts xsi:type="core:VisualizationAttribute"  name="Name" featureConcept="//@domainModel/@domainElements.8"/>
        <viewComponentParts xsi:type="core:VisualizationAttribute"  name="Price" featureConcept="//@domainModel/@domainElements.9"/>
      </viewElements>
    </interactionFlowModelElements>
\end{lstlisting}

The Model representation of this Product page structure is shown in \ref{fig:ifml-before-hierarchy-shoppingcart}

\vspace{0.5cm}
\begin{figure}[H]
  \centering
    \includegraphics[width=10cm]{images/diagrams/before/ifml-hierarchy-product.png}
  \caption{Expanded Interaction Flow Product Model Hierarchy}
  \label{fig:ifml-before-hierarchy-shoppingcart}
\end{figure}
\vspace{0.5cm}

\newpage
\subsubsection{Shared Elements and Interactions}






