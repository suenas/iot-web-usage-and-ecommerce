% !TEX root = Tesi.tex
\chapter*{Conclusions}

This work has presented a practical approach to the development of a eCommerce platform in which customer actions on physical retail stores become the basis for an improved and customised user experience.

iBeacon technologies have proven to be an asset in converting geolocalisation into detail-rich information not only on customer position but also on their behaviour and preferences. In this scenario, the Internet of Things and web-based data mining have been coupled to create a source of knowledge able to effectively inform business strategies in both virtual and physical stores.

Besides discussing which actions on the physical world would improve user experience on the virtual realm, we have also demonstrated how to apply the gathered data during the website development. By leveraging on the flexibility and scalability of the Magento platform, we were able to actually build an eCommerce website that adapts its frontend layer to each user.

In a world where user data becomes increasingly abundant and yet valuable, pervasive technologies and machine-learning processes become the tools by which brands and marketers can differentiate themselves and stand out. Offering data-driven user experiences becomes vital in increasing brand awareness and in fostering strong and long brand relationships.

The possibilities on this field are still to be explored and extended. Our goal in starting this discussion has been achieved, but further work is still to be done.


\addcontentsline{toc}{chapter}{Conclusions}
