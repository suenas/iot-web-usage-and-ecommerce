% !TEX root = Tesi.tex
\chead{}
\chapter{The combined approach}

\section{The IFML language}

The Interaction Flow Modeling Language (IFML)\cite{IFML-1, IFML-2} is designed for describing and controlling the behavior of front-end software applications, it brings several advantages to the development process such as promoting the separation of concerns between roles and increasing the overall understanding of the product for non-technical stakeholders. To achieve so IFML supports formal specification for interface composition, user interaction and event management independently of the implementation platform and it was adopted as a standard by the Object Management Group (OMG) in March 2013.

IFML supports the following concepts: 

\begin{itemize}
  \item \textbf{The view structure} describes \textit{ViewContainers}, their nesting relationships, their visibility and their reachability.
    
  \item \textbf{The view content} manages \textit{ViewComponents}, i.e., content and data entry elements contained within ViewContainers.
  
  \item \textbf{The events} defines the \textit{Events} that may affect the state of the user interface. \textit{Events} can be produced by the user’s interaction, by the application, or by an external system; 

  \item \textbf{The actions} triggered by the user’s events. The effect of an \textit{Event} is represented by an \textit{InteractionFlow} connection, which connects the event to the \textit{ViewContainer} or \textit{ViewComponent} affected by the \textit{Event}. The \textit{InteractionFlow} expresses a change of state of the user interface: the occurrence of the event triggers a change in the state that produces a transition in the user interface.
  
  \item \textbf{The navigation flow} indicates the effect of an Event on the user interface.

  \item \textbf{The data flow} indicates the data passed between \textit{ViewComponents} and \textit{Actions}
  
  \item \textbf{The parameter binding} illustrates the input-output dependencies between \textit{ViewComponents} and between \textit{ViewComponents} and \textit{Actions}. 


\end{itemize} 

\vspace{0.5cm}
\begin{figure}[htbp]
  \centering
    \includegraphics[height=10cm]{images/ifml.jpg}
  \caption{Main IFML concepts and notations.}
  \label{fig:ifml}
\end{figure}
\vspace{0.5cm}

\section{User interaction flow modeling for an eCommerce website}

eCommerce website front ends are usually built using shared and reusable components (forms, list views, detail views, etc.) which have a specific and expected behavior.
For example, Product lists and grids show record details for the user to view and interact with an action on these, "Add To Cart" call-to-action buttons are presented within product pages to trigger a different response and so on.
All these interactive operations can be represented using the IFML notation.

To demonstrate the versatility and adaptability of this modeling language we introduce a real-life example which we will use as a reference from this chapter forward: an online boutique website called \textit{"Madison Island"} specialized in fashion items running on an eCommerce platform.

Madison Island presents all the features of a typical eCommerce digital store including navigation and browsing of its catalog, product searching, customer account section, shopping cart and order processing. 


\vspace{0.5cm}
\begin{figure}[htbp]
  \centering
    \includegraphics[height=8cm]{images/home.png}
  \caption{Madison Island digital store homepage}
  \label{fig:home}
\end{figure}
\vspace{0.5cm}


Figure \ref{fig:home} shows the home page of the website. In this section, the user can select one of the product categories, access his customer area, switch the language of the website, search for an item or go directly to the shopping cart. 

In the following subsections, we analyze a few of these navigational behaviors showing their respective IFML notation representation.

\subsection{Navigational Behaviours}

\subsubsection{The Product Detail Journey}

Starting from the homepage the user can interact with the navigation menu to either select one of the product categories directly or browse the macro category page.  (Figure \ref{fig:navigation})

\vspace{0.5cm}
\begin{figure}[htbp]
  \centering
    \includegraphics[width=12cm]{images/madison/navigation.png}
  \caption{Madison Island Navigation menu}
  \label{fig:navigation}
\end{figure}
\vspace{0.5cm}

In the case of a macro category selection, the user is brought to a landing page listing all the subcategories belonging to that specific parent category (Figure \ref{fig:category-cms})

\vspace{0.5cm}
\begin{figure}[htbp]
  \centering
    \includegraphics[width=12cm]{images/madison/category-cms.png}
  \caption{Madison Island macro category page}
  \label{fig:category-cms}
\end{figure}
\vspace{0.5cm}

Alternatively, when in the case the user decides to select a specific category from the main entry point on the homepage menu he is immediately presented with a list of items belonging to that category. (Figure \ref{fig:products-list})

\vspace{0.5cm}
\begin{figure}[htbp]
  \centering
    \includegraphics[width=12cm]{images/madison/products-list.png}
  \caption{Madison Island products listing}
  \label{fig:products-list}
\end{figure}
\vspace{0.5cm}

This behavior is represented in IFML with the following scheme :
















\section{Pattern recognition}



