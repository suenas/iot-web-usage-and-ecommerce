
% !TEX root = Tesi.tex
\chapter*{Introduction}

Learn and discover user preferences by analyzing actual interactions and intelligently take advantage of those learnings, how is that obtainable?  In this thesis, we answer this question initially introducing the reader to possible sources of real usage such as the Internet of Things and Web Mining processes including software development techniques capable of modeling the acquired data (Chapter 1).
After this overall introduction, we dig into an analysis of the current state of the art for such sources of information and their related pattern recognition techniques (Chapter 2).
In Chapter 3 we expose our approach which blends both data acquisition channels presenting different use case scenarios made interacting with an actual eCommerce platform and a physical retail store for the same brand.
Starting from Chapter 4 we begin to model the data presented in the previous chapter using Model Driven Engineering notions describing metamodels and models for the domain and the data occurrences. We end the chapter illustrating a possible model transformation for the eCommerce platform based on the real usage data model so to offer a more personalized experience to the user.
After this modeling phase, we proceed with a case study on Chapter 5 introducing the eCommerce Magento platform and suggesting a possible target model serialization which would result in compatible code within the Magento ecosystem.
The last chapter covers related works, conclusions and possible evolutions for the approach presented.

\addcontentsline{toc}{chapter}{Introduction}
