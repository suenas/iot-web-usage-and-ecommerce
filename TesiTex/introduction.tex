
% !TEX root = Tesi.tex
\chapter*{Introduction}

Learning and discovering user preferences by analyzing actual interactions and intelligently take advantage of those learnings, how is that obtainable?  In Chapter 1 of this thesis, we answer this question initially introducing the reader to two possible sources of real usage data such as the Internet of Things and the Web Mining process plus the related software development techniques capable of modeling the acquired data from those.

After this general introduction, we proceed to Chapter 2 examining the current state of the art for such sources of information and their related pattern recognition techniques.

In Chapter 3 we present our approach which blends both data acquisition channels offering different use case scenarios performed interacting with an actual eCommerce platform and a physical retail store for the same brand.

Starting from Chapter 4 we begin to model the data presented in the previous chapter using Model Driven Engineering notions describing metamodels and models for the domain and the data occurrences. We conclude the chapter illustrating a possible model transformation for the eCommerce platform based on the real usage data model so to offer a more personalized experience to the user.

After the modeling phase, we proceed with a case study on Chapter 5 introducing the eCommerce Magento platform and suggesting a possible target model serialization which would result in compatible code within the Magento ecosystem.

The last chapter covers related works, conclusions and possible evolutions for the approach presented.

\addcontentsline{toc}{chapter}{Introduction}
