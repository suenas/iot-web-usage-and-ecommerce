
% !TEX root = Tesi.tex
\chapter*{Introduction}

To learn and discover user preferences by analysing actual interactions and to cleverly take advantage of that learning: How can that be achieved? In the first chapter of this thesis, we answer this question by introducing the reader to two possible sources of real usage data: the Internet of Things and the Web Mining process. In addition, we present the software development techniques used for modeling the data gathered from those sources.

Following that general introduction, we proceed to Chapter 2, in which we examine the current state-of-the art of those information sources as well as their related pattern-recognition techniques.

In Chapter 3, we discuss an approach that blends together the discussed data acquisition channels. We also offer different use-case scenarios in which an actual eCommerce platform dialogues with a physical retail store for creating a singular brand experience.

Starting from Chapter 4, we begin to model the data presented previously by using Model Driven Engineering notions and by describing metamodels and models for the domain and the data occurrences. 

We proceed then with Chapter 5, illustrating a possible model transformation for the eCommerce platform based on the real usage dynamic instance, to offer users a more customised experience.

After the modeling and processing phase, we continue with discussing a case study. Chapter 6 introduces the eCommerce Magento platform, and suggests a possible transformed model serialisation that would result in compatible code within the Magento ecosystem.

The last chapter covers related works, conclusions and a possible evolution for the presented approach.

\addcontentsline{toc}{chapter}{Introduction}
