\renewcommand{\sfdefault}{phv}
Marketing personalisation is becoming a key focus area in communication studies and development, and will draw more and more attention of marketers in the next few years. For being able to effectively personalise the experiences of their customers, brands must overcome challenges that range from learning customer behaviour as they interact with the brand itself, to creating customer experiences that are tailored according to brand knowledge of individual preferences.

If in the past brands had not enough data to understand their customers at a personal level, nowadays things are moving towards an entirely different scenario, in which big data plays a fundamental role. For the first time, companies are able to distinguish customers beyond the mere demographic dimensions, leveraging the information collected in both the virtual and the physical worlds, and detecting  and analysing patterns in their behaviour.

In this work, we describe how marketing personalisation can be achieved by coupling  the customer information resulting from web data mining with the device tracking allowed by the Internet of Things. We will apply Model Driven Engineering techniques to classify and analyse the gathered data and, from that, to deliver unique and more effective online interactions between a brand and its consumers.
